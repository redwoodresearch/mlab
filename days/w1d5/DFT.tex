\documentclass{article}
\usepackage[a4paper, total={6in, 8in}]{geometry}

\usepackage[utf8]{inputenc}
\title{\textbf{CATAM Project 17.1}}
\date{@TODO set date}

\usepackage{natbib}
% \usepackage{graphicx}
\usepackage{float}
\usepackage{amsmath} 
\usepackage{amsthm}
\usepackage{amsfonts}
\usepackage{amssymb}
\usepackage{fancyvrb}
\usepackage{breqn}
\usepackage{lmodern}
\usepackage{algorithm}
\usepackage{algpseudocode}
% \usepackage{slashbox}
\usepackage{enumitem}
\usepackage{mathtools}
\usepackage{textcomp}
% \usepackage{graphicx}
% \usepackage{subcaption}
% \usepackage[demo]{graphicx}
\usepackage[super]{nth}
% \usepackage{minted}
% \usepackage[outputdir=build]{minted}
\usepackage{minted}
\usepackage{subcaption}

\newenvironment{claim}[1]{\par\noindent\underline{Claim:}\space#1}{}
\newenvironment{claimproof}[1]{\par\noindent\underline{Proof:}\space#1}{\hfill $\square$}

\newtheorem{theorem}{Theorem}[section]
% \newtheorem{corollary}{Corollary}[theorem]
% \newtheorem{lemma}[theorem]{Lemma}
\newtheorem{lemma}[theorem]{Lemma}
\newtheorem*{definition}{Definition}
\newtheorem*{corollary}{Corollary}
\newtheorem{example}{Example}[section]

\newcommand{\pow}{\text{POW}}
\newcommand{\jac}[2]{\left( \frac{#1}{#2} \right)}
\renewcommand{\vec}[1]{\mathbf{#1}}
\renewcommand{\c}{\overline}
\newcommand{\GG}{\mathcal{G}}
\newcommand{\PP}[1]{\mathbb{P}\left( #1 \right)}
\newcommand{\FF}{\mathcal{F}}
\newcommand{\CC}{\mathbb{C}}
\newcommand{\E}[1]{\mathbb{E}\left( #1 \right)}
\newcommand{\Gnp}{\mathcal{G}(n,p)}
\renewcommand{\t}[1]{\texttt{#1}}
\newcommand{\floor}[1]{ \left\lfloor #1 \right\rfloor }
\newcommand{\ceil}[1]{ \left\lceil #1 \right\rceil }
\setcounter{secnumdepth}{0}

\begin{document}

\tableofcontents

Let $\FF$ denote the discrete Fourier transform.

[this is just a linear map $\CC^n \to \CC^n$ so it is matrix multiplication, by the matrix here: 

\texttt{https://en.wikipedia.org/wiki/Discrete_Fourier_transform\#The_unitary_DFT}.] 

Notably the matrix is invertible so the DFT is 1-to-1.

\begin{claim}
The image of $\mathbb{R}^n$ under $\FF$ is exactly the set of $x = (x_0, ... , x_{n-1}) \in \CC^n$ that satisfy 

\begin{equation}
    x_i = x_{-i}^*
\end{equation}

where the indices are taken modulo $n$.
\end{claim}

\begin{proof}
    Good maths problem.
\end{proof}

Then this means that when we take $\FF^{-1}(x)$ and know that we're getting a real sequence out, it suffices to specify (complex numbers)) $x_0, x_1, ... , x_{n/2}$, as then we can determine $x_{n/2 + 1}, ... , x_{n-1}$ from these. However, the imaginary parts of $x_0$ and $x_{n/2}$ must be 0. This is the reason for the 113 in (1,3,224,113,2), since it specifies all the information required apply $\FF^{-1}$ and get a real sequence. Moreover, I guess the entries [:,:,:,0,1] and [:,:,:,112,1] are redundant, since as mentioned for DFT to return something real these imaginary parts need be 0.

\end{document}